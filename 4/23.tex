\begingroup
\newcommand{\roto}[2]{R_{#1} \left( #2 \right)}
%
\par Let $U = \roto{x}{\theta}$.
%
In the Bloch space, this is a rotation around the x-axis by $\theta$ which, intuitively, is equivalent to:
%
$$
\roto{x}{\theta} = \roto{z}{-\frac{\pi}{2}} \roto{y}{\theta} \roto{z}{\frac{\pi}{2}}
$$
%
Using the ABC theorem:
%
\begin{verbatim}
----------------+---------------------+------------------------------
                |                     |
--[R_z(pi/2)]--[X]--[R_y(-theta/2)]--[X]--[R_z(-pi/2) R_y(theta/2)]--
\end{verbatim}
%
\par Now let $U = \roto{y}{\theta} = \roto{z}{0} \roto{y}{\theta} \roto{z}{0}$.
%
Under the ABC theorem, $C = I$:
\begin{verbatim}
---+---------------------+-------------------
   |                     |
--[X]--[R_y(-theta/2)]--[X]--[R_y(theta/2)]--
\end{verbatim}
%
\endgroup
